\documentclass[12pt,letterpaper]{article}
\usepackage{amsmath, amssymb, graphicx, fullpage, paralist, algorithmic}
\begin{document}
{\noindent\bf Solution to} Homework 1\newline
{\bf Group members' netid:} rjones27, jharris39, khan14 \newline
\hrule height 3pt \bigskip

\begin{enumerate}
\item {\bf Problem 1: The Closet Mayhem Problem}

\begin{compactenum}[(A)]
\item Don't Have!
\newline

\item Don't Have!
\newline

\item Don't Have!
\newline

\item Don't Have!
\newline

\item Assuming $X$ is a strongly connected component of Graph $G$ and $X$ is a sink then $X'$ (the mirror $X$) is not a source. \newline
Given $X$ has no outgoing edges because it is a sink and $X'$ has at least 1 incoming edge let $[i:x_i]$ and $[i:x'_i]$ be two complementing nodes for the $i$th broom outside of $X$ and $X'$. Now let $[i:X_i]$ force some node in $X'$. Therefore by the proof done in problem B we know $X$ must have a edge between the node $[i:x'_i]$ and the complement of node $[i:x_i]$ and $X'$ to each other. \newline
This outgoing edge from $X$ to $[i:x'_i]$ makes $X$ not a sink and therefore contradicts the hypothesis. So if $X$ is a sink the $X'$ has to be a source.\newline

\item Don't Have!

\end{compactenum}

\item {\bf Problem 2: Heavy Time}
\begin{compactenum}[(A)]
\item Check each node if it is a sink if so then add an edge to the target sink:\newline

\begin{algorithmic}
\FOR{each node in the adjacency list}
	\IF{$\#$ of out going edges = 0}
		\STATE {Add edge to $s$}
	\ENDIF
\ENDFOR
\end{algorithmic}

Running Time: $O(n)$ \newline
Number of Edges Worst Case: $m$ \newline

\item First, reverse the graph and topological sort the the vertices.
Then, use DFS to find all the vertices reachable from s and remove the vertices that are not.

\begin{algorithmic}
\FOR{each vertex v}
  \STATE $v.t_max = 0$
\ENDFOR
\FOR{each vertex v in the topological order}
  \IF{v is significant}
    \STATE $v.t_{max} = t_{max}+1$
  \ENDIF
  \FOR{each edge (v,u)}
    \IF{$v.t_{max} > u.t_{max}$}
      \STATE $u.t_{max} = v.t_{max}$
    \ENDIF
  \ENDFOR
\ENDFOR
\end{algorithmic}
Then, any vertex with a $t_{max} > t$ can reach s via a path going through t significant vertices.\newline

\item Our algorithm is to negate all the weights on the edges, then use the Bellman-Ford algorithm.
Let $G'$ be $G$ with all the edge weights negated.
Since there are no cycles in $G$, there will be no negative cycles; therefore, the Bellman-Ford algorithm will find the shortest distance between $v$ to $s$ in our modified graph, $G'$.
The weight of every path will be the opposite of what it was for $G$, so the lightest path in $G'$ will be the heaviest in $G$.
Negating every edge weight require $O(m)$ time, and the Bellman-Ford algorithm takes $O(nm)$, so this algorithm runs in $O(nm)$. \newline

\item First, topological sort the the vertices.
If there is a path that visits all of the vertices, it will be the only topological sorting.

Proof: Suppose there are multiple topological sortings. 
There must be at least two vertices that could be in a different order.
Let A and B be two such vertices.
Then there must be no path from A to B nor from B to A.
If there was a path from A to B, then A would have to come before B in the sort order.
If there was a path from B to A, then B would have to come before A in the sort order.
So, there is no path that goes through both A and B, so there is no path that visits every vertex.

To find the path that visits all vertices, start at the source (first in the topological sort).
Then, from each node, go to the next one in the topological sorting.
The edges traversed will form the path that visits all the vertices, if it exists.\newline
\end{compactenum}

\item {\bf Wishful Graph}
\begin{compactenum}[(A)]
\item $R$ is an equivalence relation.
$R$ is defined as $uRv$ iff ($u$ can reach $v$) or ($v$ can reach $u$)
Clearly, $u$ and $v$ can be interchanged thus it is symmetric.\newline
If $aRb$ and $bRc$ then $a$ can reach $b$ and $b$ can reach $c$ by the definition. So then $a$ must be able to reach $c$ so $aRc$ given $aRb$ and $bRc$, therefore $R$ is transitive.\newline
Since a node can always reach itself we know that $aRa$ therefore $R$ is reflexive. \newline
Since $R$ is symmetric, transitive, and reflexive we know $R$ is an equivalence relation.\newline

\item Our algorithm was to first run the {\bf Undirected Simple SCC} on $G$. Then run a {\bf Topological Sort} on $G$ which should only work if it is a DAG and thus not uselessly connected. If it is false then it is uselessly connected. (Note: since both Algorithms are defined in class we did not write the algorithms).

\end{compactenum}
\end{enumerate}
\end{document}  



